\documentclass[10pt,a4paper]{article}
\usepackage[utf8]{inputenc}
\usepackage{amsmath}
\usepackage{amsfonts}
\usepackage{amssymb}
\usepackage{graphicx}
\usepackage[swedish]{babel}
\usepackage[utf8]{inputenc}

\graphicspath{}

\author{
  \texttt{Sebastian Bångerius}
}

\begin{document}
\pagenumbering{gobble}

\title{Saker vi ska kunna till tentorna}
\maketitle

\cleardoublepage

\tableofcontents

\clearpage
\section{Disclaimer}
Även om detta dokument ser ut som något MAI skickar ut så är det inte det. Jag har sett över det mesta men fel kan förekomma. Jag kan inte lova att allt detta räcker för att klara tentorna.

\section{Flervariabelanalys}
Vi har tagit upp områden såsom \textit{Flervariabelgränsvärden}, \textit{Partialderivator}, \textit{Partiella differetialekvationer}, \textit{Kedjeregeln} , \textit{}, \textit{Gradienter}, \textit{Funktionaldeterminanter} och \textit{Dubbel-/Trippelintegraler} samt \textit{Medelvärdessatsen}

\subsection{Triangelolikheten}
\begin{equation}
\left|x+y\right|\leq\left|x\right|+\left|y\right|
\end{equation}
\begin{flushleft}
Triangelolikheten behandlar vektorer av olika grader. Kan effektivt användas för att stänga in flervarabelfunktioner.
\end{flushleft}
\begin{center}
\includegraphics[scale=0.5]{triangelolikhet}
\end{center}

\subsection{Kedjeregeln}
Kedjeregeln används för att kunna bestämma derivator vid variabelbyten. I formuleringen nedan är $f$ en funktion av $x$ och $y$ enligt $f(x(u,v),y(u,v))$. Tänk på att t.ex. $x=v$ \underline{ej} medför att $\frac{\partial f}{\partial x} = \frac{\partial f}{\partial v}$.
\begin{equation}
\frac{\partial f}{\partial u}=\frac{\partial f}{\partial x}\cdot \frac{\partial x}{\partial u} + \frac{\partial f}{\partial y} \cdot \frac{\partial y}{\partial u}
\end{equation}
\begin{equation}
\frac{\partial f}{\partial v}=\frac{\partial f}{\partial x}\cdot \frac{\partial x}{\partial v} + \frac{\partial f}{\partial y} \cdot \frac{\partial y}{\partial v}
\end{equation}

\subsection{Funktionaldeterminant}
Vid variabelbyte i en integral behöver man multiplicera funktionen som skall integreras med funktionaldeterminanten. Detta kompenserar upp eventuella utsträckningar etc. som uppstått till följd av t.ex. brist på ortogonalitet i den nya basen. detta görs på följande sätt för ett variabelbyte $f(x,y) \rightarrow \tilde{f}(u,v)$.
\begin{equation}
\iint_D f(x,y)\,dx\,dy = \iint_E \tilde{f}(u,v)\cdot \frac{d(x,y)}{d(u,v)}\,du\,dv
\end{equation}
där funktionaldeterminanten $\frac{d(x,y)}{d(u,v)}$ räknas ut enligt

\begin{equation}
\frac{d(x,y)}{d(u,v)} = \left| \begin{array}{ccc}\frac{\partial x}{\partial u} & \frac{\partial x}{\partial v} \\ \frac{\partial y}{\partial u} & \frac{\partial y}{\partial v} \end{array} \right| \neq 0
\end{equation}
Minns även att 
\begin{equation}
\frac{d(x,y)}{d(u,v)} = \frac{1}{\frac{d(u,v)}{d(x,y)}}
\end{equation}
Viktiga funktionaldeterminanter är t.ex. $\frac{d(x,y)}{d(\rho,\phi)} = \rho$ (cylindriska koordinater, du vet vad $x(\rho,\phi)$ och $y(\rho,\phi)$ är) och $\frac{d(x,y,z)}{d(\rho,\theta,\phi)} = \rho^2 sin(\theta)$ (Sfäriska koordinater där $\left\{ \begin{array}{lll}
x=\rho sin(\theta)cos(\phi)\\
y=\rho sin(\theta)sin(\phi)\\
z=\rho cos(\theta)
\end{array} \right. $). Dessa två får användas utan härledning på tentan


\subsection{Gradient}
Gradient beskriver vi oftast som en vektor som i en viss punkt pekar åt det håll som en funktionsyta $f(x,y)$ lutar mest (uppåt). Se pilarna i botten av bild nedan.
\begin{center}
\includegraphics[scale=0.5]{gradient}
\end{center}
att räkna ut gradienten är ganska 'straight forward'. Den ges (med denna kurs notering) av
\begin{equation}
\nabla f = (f'_x,f'_y)
\end{equation}

\subsection{Differentierbarhet}
En funktion $f(x_1,x_2,...,x_n)$ är differenterbar i en punkt $(a_1,a_2,...,a_n)$ om man oavsett vilket håll man närmar sig ifrån får en väldefinierad derivata i punkten.

\subsection{Gränsvärden}
När man skall bestämma ett flervariabelgränsvärde får man problemet att funktionen beror på flera variabler och således kan man närma sig en punkt från oändligt många håll. Detta löses genom att stänga in funktionen mellan två funktioner som endast beror av en variabel. Ett bra sätt att på ett beroende av endast en variabel är att t.ex. byta till polära eller cylindriska koordinater.

\section{Elektromagnetism}
Fält, fält, fält, fält.
\subsection{Jag tar livet av mig}
Kan inte den här kursen. Snälla någon annan skriv.

\section{Standard (minns från envar)}

\subsection{Standardgräsvärden}
\begin{equation}
\lim_{x\to 0} \frac{\ln(1+x)}{x}=1
\end{equation}
\begin{equation}
\lim_{x\to 0} \frac{e^x-1}{x}=1
\end{equation}
\begin{equation}
\lim_{x\to 0} \frac{\sin(x)}{x}=1
\end{equation}
\begin{equation}
\lim_{x\to 0+} x^\alpha \ln(x)=0
\end{equation}
\begin{equation}
\lim_{x\to \infty} \frac{\ln(x)}{x^\alpha}=0
\end{equation}
\begin{equation}
\lim_{x\to \infty} \frac{x^\alpha}{a^x}=0
\end{equation}
För ovanstående samband gäller att $\alpha \in \mathbb{R}$ , $\alpha >0$ och $a \in \mathbb{R}$ , $a > 1$

\subsection{Stardardprimitiver}
Glöm inte lägga till konstant $C$.
\begin{equation}
\int \alpha \,dx = \alpha x \hspace{10mm}(\alpha\in\mathbb{R})
\end{equation}
\begin{equation}
\int x^\alpha \,dx = \frac{a^{\alpha+1}}{\alpha+1} \hspace{10mm}(\alpha\in\mathbb{R},\alpha\neq-1)
\end{equation}
\begin{equation}
\int \frac{1}{x} \,dx = \ln|x|
\end{equation}
\begin{equation}
\int e^x \,dx = e^x
\end{equation}
\begin{equation}
\int \cos(x) \,dx = \sin(x)
\end{equation}
\begin{equation}
\int sin(x) \,dx = -\cos(x)
\end{equation}
\begin{equation}
\int \frac{1}{1+x^2} \,dx = \arctan(x)
\end{equation}
\begin{equation}
\int \frac{1}{\sqrt{1-x^2}} \,dx = \arcsin(x)
\end{equation}
\begin{equation}
\int \frac{1}{\cos^2(x)} \,dx = tan(x)
\end{equation}
\begin{equation}
\int \frac{1}{\sin^2(x)} \,dx = -\cot(x)
\end{equation}
\begin{equation}
\int \frac{1}{\sqrt{x^2+\alpha}} \,dx = \ln|x+\sqrt{x^2+\alpha}| \hspace{10mm} (\alpha
\in \mathbb{R},\alpha\neq0)
\end{equation}


\end{document}